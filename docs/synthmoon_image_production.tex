\documentclass[12pt]{article}
\usepackage[margin=1in]{geometry}
\usepackage[T1]{fontenc}
\usepackage[utf8]{inputenc}
\usepackage{lmodern}
\usepackage{amsmath}
\usepackage{amssymb}
\usepackage{hyperref}
\usepackage{enumitem}
\usepackage{graphicx}
\usepackage{datetime}
\renewcommand{\dateseparator}{-}
\settimeformat{hhmmsstime}

\setlist{nosep}

\title{synthmoon: How Images Are Produced\\(Legacy and DEM-Advanced Modes)}
\author{Peter Thejll}
\date{\today\ \currenttime}

\begin{document}
\maketitle

\section{Purpose}
This note documents the current image-generation pipeline in \texttt{synthmoon}, with emphasis on:
\begin{itemize}
  \item the legacy simple setup (point Sun + point Earth),
  \item the advanced setup (extended-source illumination + DEM-aware geometry/shadowing),
  \item the movie-generation scripts.
\end{itemize}

\section{Top-Level Pipeline}
The main renderer is \texttt{synthmoon/run\_v0.py}. For one UTC timestamp, it does:
\begin{enumerate}
  \item Load SPICE kernels and resolve the Moon-fixed frame (e.g.\ \texttt{MOON\_ME}).
  \item Compute Sun/Earth/Moon/observer geometry in J2000.
  \item Build camera rays and intersect with a lunar sphere.
  \item Optionally refine intersections using a lunar DEM (\texttt{dem\_refine\_iter}).
  \item Compute local normal, slope, elevation, and lunar albedo.
  \item Compute solar I/F (\texttt{lambert} or \texttt{hapke}; point or extended Sun).
  \item Compute Earthshine I/F (point-Earth legacy or extended-disk Earth quadrature).
  \item Sum contributions:
  \[
    I/F_{\text{total}} = I/F_{\odot} + I/F_{\oplus}.
  \]
  \item Convert to radiance using:
  \[
    L = (I/F)\,\frac{F_{\odot,\text{Moon}}}{\pi}.
  \]
  \item Optionally downsample from high render resolution (e.g.\ 2048) to output (e.g.\ 512) by NaN-aware block mean.
  \item Write FITS layers and metadata.
\end{enumerate}

\section{CLI Arguments (Complete)}
\subsection{Main renderer: \texttt{uv run -m synthmoon.run\_v0}}
\begin{itemize}
  \item \texttt{--config PATH}: path to config file (default: \texttt{scene.toml}).
  \item \texttt{--utc ISO8601}: override \texttt{time.utc} from config.
  \item \texttt{--out PATH}: override \texttt{paths.out\_fits} from config.
  \item \texttt{--only-layer-index N}: write only layer \texttt{N} (1-based).
  \item \texttt{--comparison-child}: internal flag for spawned comparison renders.
  \item \texttt{--legacy-parallel}: internal legacy mode (point Sun + point Earth).
\end{itemize}

\subsection{Video builder: \texttt{uv run python tools/build\_movie\_video\_hourly.py}}
\begin{itemize}
  \item \texttt{--config PATH} (default: \texttt{scene.toml})
  \item \texttt{--start-utc ISO8601} (required)
  \item \texttt{--hours N} (default: \texttt{672})
  \item \texttt{--step-hours N} (default: \texttt{1})
  \item \texttt{--fps N} (default: \texttt{24})
  \item \texttt{--out-mp4 PATH} (default: \texttt{OUTPUT/movie\_28d\_hourly\_iftotal.mp4})
  \item \texttt{--out-mov PATH} (default: \texttt{OUTPUT/movie\_28d\_hourly\_iftotal.mov})
  \item \texttt{--workdir PATH} (default: \texttt{/tmp/synthmoon\_movie\_frames})
  \item \texttt{--tmp-fits PATH} (default: \texttt{/tmp/synthmoon\_movie\_frame.fits})
  \item \texttt{--vmin FLOAT} (default: \texttt{0.0})
  \item \texttt{--vmax FLOAT} (default: auto from first frame p99.9)
  \item \texttt{--crf N} (default: \texttt{18})
\end{itemize}

\subsection{Movie-cube builder: \texttt{uv run python tools/build\_movie\_cube\_hourly.py}}
\begin{itemize}
  \item \texttt{--config PATH} (default: \texttt{/tmp/scene\_movie.toml})
  \item \texttt{--start-utc ISO8601} (required)
  \item \texttt{--hours N} (default: \texttt{672})
  \item \texttt{--step-hours N} (default: \texttt{1})
  \item \texttt{--out PATH} (default: \texttt{OUTPUT/movie\_28d\_hourly\_iftotal.fits})
  \item \texttt{--tmp-frame PATH} (default: \texttt{/tmp/synthmoon\_movie\_frame.fits})
  \item \texttt{--dtype \{float32,float64\}} (default: \texttt{float32})
\end{itemize}

\section{Legacy Simple Setup}
Legacy mode is activated internally by the comparison workflow (\texttt{--legacy-parallel}) and is defined as:
\begin{itemize}
  \item \textbf{Sun}: point source (\texttt{sun.extended\_disk = false})
  \item \textbf{Earthshine}: point source (\texttt{earth.point\_source = true})
  \item \textbf{Earthshine enabled}: \texttt{illumination.include\_earthlight = true}
\end{itemize}

So this is not ``no Earthshine''; it is a physically simpler Earthshine approximation.

\subsection{Legacy point-Earth model}
In \texttt{illumination.py}, \texttt{earthlight\_if\_point\_source(...)} uses one arrival direction (Earth center) and a Lambert-sphere phase factor:
\[
\Phi(g)=\frac{\sin g + (\pi-g)\cos g}{\pi}.
\]
The irradiance on a Moon facet scales as
\[
E_{\oplus} \propto \frac{2}{3} A_{\oplus}\,F_{\odot,\oplus}\,\Phi(g)\left(\frac{R_{\oplus}}{d_{\oplus m}}\right)^2 \mu_m,
\]
and then converts to lunar I/F with local lunar albedo and solar-normalization factor.

\section{Advanced Setup With DEM}
The advanced mode in \texttt{scene.toml} typically uses:
\begin{itemize}
  \item \textbf{Extended Sun}: \texttt{sun.extended\_disk = true}
  \item \textbf{Extended Earth}: \texttt{earth.point\_source = false}, disk quadrature enabled
  \item \textbf{DEM}: \texttt{moon.dem\_fits} set, with \texttt{dem\_refine\_iter > 0}
  \item \textbf{DEM Sun shadowing}: \texttt{shadows.sun = "dem"}
\end{itemize}

\subsection{DEM geometry refinement}
Ray--Moon intersections start from a sphere, then iterate with per-hit DEM radius
\[
R(\lambda,\phi)=R_0+\texttt{dem\_scale}\left(R_{\text{DEM}}(\lambda,\phi)-R_0\right),
\]
which improves hit points and local topographic normals.

\subsection{Extended Sun + DEM occlusion}
For extended Sun, \texttt{run\_v0.py} estimates visible solar-disk fraction per pixel:
\[
f_{\text{vis},\odot}\in[0,1].
\]
It samples directions over the Sun disk and ray-tests occlusion against DEM topography. The direct-sun term is scaled by
\[
I/F_{\odot,\text{final}} = f_{\text{vis},\odot}\,I/F_{\odot,\text{extended}}.
\]
Diagnostic output layers \texttt{SUNVIS} and \texttt{SUNBLK} store visible and blocked fractions.

\subsection{Extended Earthshine (advanced Earth model)}
\texttt{earthlight\_if\_tilecached(...)} performs disk quadrature over Earth, including:
\begin{itemize}
  \item Earth phase and Sun illumination of Earth patches,
  \item visibility of Earth patches toward Moon,
  \item optional Earth albedo map (\texttt{DATA/earth\_albedo.fits}),
  \item optional simple cloud and ocean-glint terms,
  \item tile caching (\texttt{earthlight\_tile\_px}) for speed.
\end{itemize}

\section{Output Layers and Products}
When writing full cubes, the renderer writes layers including:
\texttt{IFTOTAL}, \texttt{IF\_SUN}, \texttt{IF\_EARTH}, \texttt{RADTOT}, \texttt{RAD\_SUN}, \texttt{RAD\_EAR},
\texttt{ALBMOON}, \texttt{ELEV\_M}, \texttt{SLOPDEG}, \texttt{SUNVIS}, \texttt{SUNBLK}, \texttt{SELON}, \texttt{SELAT}.

For long runs, a single 2D layer can be written (\texttt{write\_cube=false} + \texttt{only\_layer\_index}).
Current operational setup writes only \texttt{IFTOTAL} as a 2D floating-point FITS.

\section{Comparison Mode (Advanced vs Legacy)}
With \texttt{[comparison] enabled=true}, one command produces:
\begin{itemize}
  \item \texttt{\_advanced.fits}
  \item \texttt{\_legacy\_parallel.fits}
  \item \texttt{\_diff\_if.fits} (if \texttt{write\_diff=true})
\end{itemize}
Diff layers include absolute and relative metrics (\texttt{DIFF\_IF}, \texttt{PCTDIFF}, \texttt{PCTROB}, etc.), plus explicit source layers \texttt{ADV\_IF} and \texttt{LEG\_IF}.

\section{Movie Maker Scripts}
Two scripts are relevant:
\begin{itemize}
  \item \texttt{tools/build\_movie\_video\_hourly.py}: renders frames by repeatedly calling \texttt{run\_v0}, converts each frame to 16-bit PNG, and encodes MP4/MOV with \texttt{ffmpeg}. This avoids huge FITS cubes.
  \item \texttt{tools/build\_movie\_cube\_hourly.py}: builds a FITS time cube (\texttt{NFRAMES} $\times$ ny $\times$ nx).
\end{itemize}

\subsection{Typical video command}
\begin{verbatim}
uv run python tools/build_movie_video_hourly.py \
  --config /tmp/scene_movie_hires.toml \
  --start-utc 2006-02-13T06:18:45Z \
  --hours 240 --step-hours 6 --fps 24 \
  --out-mp4 OUTPUT/movie_60d_6h_iftotal.mp4 \
  --out-mov OUTPUT/movie_60d_6h_iftotal.mov
\end{verbatim}

\section{Practical Notes}
\begin{itemize}
  \item High-quality 512x512 output is produced by render-then-downsample (e.g.\ 2048 $\rightarrow$ 512), not direct low-res rendering.
  \item FITS headers store run provenance (\texttt{RUNMODE}, \texttt{SUNEXT}, \texttt{EARTHPT}, \texttt{SHSUN}, DEM metadata, observer location, etc.).
  \item The project has moved from the baseline (\texttt{0.2.0}) to comparison support (\texttt{0.3.0}) and DEM-aware extended-Sun shadow correction (\texttt{0.3.1}).
\end{itemize}

\begin{figure}
  \centering
  \includegraphics[width=0.99\linewidth]{fig/compare_dots_vs_dem_20060217T061845Z_side_by_side.png}
  \caption{Synthetic lunar I/F images at the same UTC (2006-02-17T06:18:45Z), observer geometry, and output scale. Left: advanced model with LOLA DEM topography, DEM-aware solar shadowing, and extended-source Sun and Earthshine. Right: legacy dot-source model with Sun and Earth treated as point sources and no DEM terrain model. Both panels are rendered at high resolution and downsampled to $512\times512$.}
  \label{fig:sidebyside}
\end{figure}

\end{document}
